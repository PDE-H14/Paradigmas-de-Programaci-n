\section{Subir archivos a un repositorio}
	\begin{itemize}
		\item[\textbf{\texttt{1.-}}] Abrir una terminal.
		\item[\textbf{\texttt{2.-}}] Dirigirse a la carpeta del repositorio.
		\item[\textbf{\texttt{3.-}}] Para agregar archivos al repositorio, deben estar en la misma carpeta, luego ingresamos el siguiente comando
		\begin{lstlisting}[language=bash, caption= Ejemplo. Agregar un archivo]
$ git add nombreDelArchivo.extension\end{lstlisting}
		o si se desean agregar todos los archivos
		\begin{lstlisting}[language=bash, caption= Ejemplo. Agregar todos los archivos]
$ git add .\end{lstlisting}
		\item[\textbf{\texttt{4.-}}] Para verificar los archivos que se van a subir se puede ejecutar el comando 
		\begin{lstlisting}[language=bash, caption= Ejemplo. Estatus del almacenaje]
$ git status\end{lstlisting}
		\item[\textbf{\texttt{5.-}}] Ahora hay que ejecutar el siguiente comando, con una breve descripción de las novedades que se van a subir
		\begin{lstlisting}[language=bash, caption= Ejemplo. Hacer commit]
$ git commit -m "Breve Descripcion"\end{lstlisting}
		\item[\textbf{\texttt{6.-}}] Finalmente, correr el comando 
		\begin{lstlisting}[language=bash, caption= Ejemplo. Subir al repositorio]
$ git push\end{lstlisting}
		\item[\textbf{\texttt{7.-}}] Posteriormente, en dos pasos, ingresar el nombre de usuario y el Token que se generó. 
		\begin{lstlisting}[language=bash, caption= Ejemplo. Subir al repositorio]       
Username for 'https://github.com': nombreDeUsuario
Password for 'https://user@github.com': token\end{lstlisting} 
	\end{itemize}